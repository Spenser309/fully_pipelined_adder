\documentclass[10pt,letterpaper]{article}
\usepackage[latin1]{inputenc}
\usepackage{amsmath}
\usepackage{amsfonts}
\usepackage{amssymb}
\usepackage{url}
\author{Spenser Gilliland\\
\url{Spenser309@gmail.com}\\
ID:A20254941}
\title{Final Project Proposal: Quantitative Evaluation of RTL Power Reduction Techniques}

\begin{document}

\maketitle
\bigskip

\begin{abstract}
Power reduction of integrated circuits is increasingly important due to current heat dissipation and battery capacity limits.  This project utilizes a number of techniques to reduce the power consumption of a pipelined adder.  Specifically, glitch reduction, operand isolation, enhanced clock gating and precomputation logic are utilized. The adder is a custom design already implemented in verilog available on the web from \cite{code}.
\end{abstract}

\pagebreak

\section{Objective}
The objective of this project is to quantify the effect on power reduction of the various low power techniques.  The project is based on my design of a pipelined 32 bit adder implemented in Verilog.

\section{Work Plan}
There will be a total of 6 stages from which artifacts will be generated.

\begin{center}
\begin{tabular}{|c|c|}
\hline Stages & Description \\
\hline 
\hline Stage 0 & Benchmarks of the original circuit. \\ 
\hline Stage 1 & Benchmarks after enhanced clock gating \\ 
\hline Stage 2 & Benchmarks after glitch reduction \\ 
\hline Stage 3 & Benchmarks after operand isolation \\ 
\hline Stage 4 & Benchmarks after precomputation logic \\ 
\hline Stage 5 & Benchmarks utilizing all beneficial techniques \\ 
\hline 
\end{tabular} 
\end{center}

For each stage, a patch file will be provided.  A patch file provides a difference of the source files of the design.

In order to determine the relative merits of each of these techniques the following metrics will be recorded and analyzed.

\begin{center}
\begin{tabular}{|c|}
\hline Metrics \\
\hline 
\hline Total Power \\
\hline Maximum Frequency \\
\hline Total Area \\
\hline 
\end{tabular}
\end{center}


\section{Testbench}

The testbench will be an exhaustive test bench of all reasonable combinations of the circuit.  In all possible pipeline stages.  This will be an enhancement to the current test bench.

\section{Expected Results}

I expect that each technique will reduce the power consumption of the pipelined adder and the overall design will be increased in quality.

\begin{thebibliography}{9}

\bibitem{requirements} ECE 530 High Performance VLSI/IC Systems Term Project (Fall 2011)
\bibitem{code} \url{https://github.com/Spenser309/fully_pipelined_adder/}

\end{thebibliography}

\end{document}